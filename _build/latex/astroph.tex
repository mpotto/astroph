%% Generated by Sphinx.
\def\sphinxdocclass{report}
\documentclass[letterpaper,10pt,english]{sphinxmanual}
\ifdefined\pdfpxdimen
   \let\sphinxpxdimen\pdfpxdimen\else\newdimen\sphinxpxdimen
\fi \sphinxpxdimen=.75bp\relax

\PassOptionsToPackage{warn}{textcomp}
\usepackage[utf8]{inputenc}
\ifdefined\DeclareUnicodeCharacter
% support both utf8 and utf8x syntaxes
  \ifdefined\DeclareUnicodeCharacterAsOptional
    \def\sphinxDUC#1{\DeclareUnicodeCharacter{"#1}}
  \else
    \let\sphinxDUC\DeclareUnicodeCharacter
  \fi
  \sphinxDUC{00A0}{\nobreakspace}
  \sphinxDUC{2500}{\sphinxunichar{2500}}
  \sphinxDUC{2502}{\sphinxunichar{2502}}
  \sphinxDUC{2514}{\sphinxunichar{2514}}
  \sphinxDUC{251C}{\sphinxunichar{251C}}
  \sphinxDUC{2572}{\textbackslash}
\fi
\usepackage{cmap}
\usepackage[T1]{fontenc}
\usepackage{amsmath,amssymb,amstext}
\usepackage{babel}



\usepackage{times}
\expandafter\ifx\csname T@LGR\endcsname\relax
\else
% LGR was declared as font encoding
  \substitutefont{LGR}{\rmdefault}{cmr}
  \substitutefont{LGR}{\sfdefault}{cmss}
  \substitutefont{LGR}{\ttdefault}{cmtt}
\fi
\expandafter\ifx\csname T@X2\endcsname\relax
  \expandafter\ifx\csname T@T2A\endcsname\relax
  \else
  % T2A was declared as font encoding
    \substitutefont{T2A}{\rmdefault}{cmr}
    \substitutefont{T2A}{\sfdefault}{cmss}
    \substitutefont{T2A}{\ttdefault}{cmtt}
  \fi
\else
% X2 was declared as font encoding
  \substitutefont{X2}{\rmdefault}{cmr}
  \substitutefont{X2}{\sfdefault}{cmss}
  \substitutefont{X2}{\ttdefault}{cmtt}
\fi


\usepackage[Bjarne]{fncychap}
\usepackage{sphinx}

\fvset{fontsize=\small}
\usepackage{geometry}

% Include hyperref last.
\usepackage{hyperref}
% Fix anchor placement for figures with captions.
\usepackage{hypcap}% it must be loaded after hyperref.
% Set up styles of URL: it should be placed after hyperref.
\urlstyle{same}
\addto\captionsenglish{\renewcommand{\contentsname}{Contents:}}

\usepackage{sphinxmessages}
\setcounter{tocdepth}{1}



\title{astroph}
\date{Dec 07, 2019}
\release{v0.0}
\author{Mateus Piovezan Otto}
\newcommand{\sphinxlogo}{\vbox{}}
\renewcommand{\releasename}{Release}
\makeindex
\begin{document}

\pagestyle{empty}
\sphinxmaketitle
\pagestyle{plain}
\sphinxtableofcontents
\pagestyle{normal}
\phantomsection\label{\detokenize{index::doc}}



\chapter{Getting started to \sphinxstyleemphasis{Sphinx}}
\label{\detokenize{usage/installation:getting-started-to-sphinx}}\label{\detokenize{usage/installation::doc}}
The syntax is pretty much like \sphinxstyleemphasis{Markdown}. Are them related?


\section{Documenting objects}
\label{\detokenize{usage/installation:documenting-objects}}\index{enumerate() (built-in function)@\spxentry{enumerate()}\spxextra{built-in function}}

\begin{fulllineitems}
\phantomsection\label{\detokenize{usage/installation:enumerate}}\pysiglinewithargsret{\sphinxbfcode{\sphinxupquote{enumerate}}}{\emph{sequence}\sphinxoptional{, \emph{start=0}}}{}
Return an iterator that yields tuples of an index and an item of the
\sphinxstyleemphasis{sequence}. (And so on.)

\end{fulllineitems}


the {\hyperref[\detokenize{usage/installation:enumerate}]{\sphinxcrossref{\sphinxcode{\sphinxupquote{enumerate()}}}}} function can be used for …

Referencing \sphinxcode{\sphinxupquote{io.open()}}


\subsection{Testing some headings}
\label{\detokenize{usage/installation:testing-some-headings}}\index{CIFParser (class in xtalphases.data.preprocess)@\spxentry{CIFParser}\spxextra{class in xtalphases.data.preprocess}}

\begin{fulllineitems}
\phantomsection\label{\detokenize{usage/installation:xtalphases.data.preprocess.CIFParser}}\pysiglinewithargsret{\sphinxbfcode{\sphinxupquote{class }}\sphinxcode{\sphinxupquote{xtalphases.data.preprocess.}}\sphinxbfcode{\sphinxupquote{CIFParser}}}{\emph{filename, cif\_columns={[}'index\_h', 'index\_k', 'index\_l', 'FOBS', 'SIGFOBS', 'UOBS', 'SIGUOBS', 'FC', 'PHI', 'FOM', 'RESOL', 'pdbx\_r\_free\_flag'{]}}}{}~\index{get\_pdb\_name() (xtalphases.data.preprocess.CIFParser method)@\spxentry{get\_pdb\_name()}\spxextra{xtalphases.data.preprocess.CIFParser method}}

\begin{fulllineitems}
\phantomsection\label{\detokenize{usage/installation:xtalphases.data.preprocess.CIFParser.get_pdb_name}}\pysiglinewithargsret{\sphinxbfcode{\sphinxupquote{get\_pdb\_name}}}{}{}
Get PDB\_ID of the file being parsed.
\begin{quote}\begin{description}
\item[{Returns}] \leavevmode
\sphinxstylestrong{name}

\item[{Return type}] \leavevmode
PDB\_ID of the file being parsed.

\end{description}\end{quote}

\end{fulllineitems}

\index{header\_parse() (xtalphases.data.preprocess.CIFParser method)@\spxentry{header\_parse()}\spxextra{xtalphases.data.preprocess.CIFParser method}}

\begin{fulllineitems}
\phantomsection\label{\detokenize{usage/installation:xtalphases.data.preprocess.CIFParser.header_parse}}\pysiglinewithargsret{\sphinxbfcode{\sphinxupquote{header\_parse}}}{}{}
Parse numeric and textual info on .cif header.

\end{fulllineitems}

\index{header\_refln\_df() (xtalphases.data.preprocess.CIFParser method)@\spxentry{header\_refln\_df()}\spxextra{xtalphases.data.preprocess.CIFParser method}}

\begin{fulllineitems}
\phantomsection\label{\detokenize{usage/installation:xtalphases.data.preprocess.CIFParser.header_refln_df}}\pysiglinewithargsret{\sphinxbfcode{\sphinxupquote{header\_refln\_df}}}{\emph{headercols=None}, \emph{reflncols=None}, \emph{phierror=False}}{}
Convert header and reflection into a single DataFrame. Header
and reflections columns can be specified via headercols and
reflncols arguments.
\begin{quote}\begin{description}
\item[{Parameters}] \leavevmode\begin{itemize}
\item {} 
\sphinxstyleliteralstrong{\sphinxupquote{headercols}} (\sphinxstyleliteralemphasis{\sphinxupquote{header columns to be included in the final DataFrame.}}) \textendash{} 

\item {} 
\sphinxstyleliteralstrong{\sphinxupquote{reflncols}} (\sphinxstyleliteralemphasis{\sphinxupquote{reflection columns to be included in the final DataFrame.}}) \textendash{} 

\end{itemize}

\item[{Returns}] \leavevmode
\sphinxstylestrong{header\_refln\_df}

\item[{Return type}] \leavevmode
DataFrame containing both header and reflection info.

\end{description}\end{quote}

\end{fulllineitems}

\index{header\_to\_df() (xtalphases.data.preprocess.CIFParser method)@\spxentry{header\_to\_df()}\spxextra{xtalphases.data.preprocess.CIFParser method}}

\begin{fulllineitems}
\phantomsection\label{\detokenize{usage/installation:xtalphases.data.preprocess.CIFParser.header_to_df}}\pysiglinewithargsret{\sphinxbfcode{\sphinxupquote{header\_to\_df}}}{\emph{columns=None}}{}
Convert header dictionary to a pandas DataFrame.
\begin{quote}\begin{description}
\item[{Returns}] \leavevmode
\begin{itemize}
\item {} 
\sphinxstylestrong{cifheader\_df} (\sphinxstyleemphasis{pandas DataFrame containing all information})

\item {} 
\sphinxstyleemphasis{stored on the header dictionary.}

\end{itemize}


\end{description}\end{quote}

\end{fulllineitems}

\index{header\_to\_series() (xtalphases.data.preprocess.CIFParser method)@\spxentry{header\_to\_series()}\spxextra{xtalphases.data.preprocess.CIFParser method}}

\begin{fulllineitems}
\phantomsection\label{\detokenize{usage/installation:xtalphases.data.preprocess.CIFParser.header_to_series}}\pysiglinewithargsret{\sphinxbfcode{\sphinxupquote{header\_to\_series}}}{\emph{columns=None}}{}
Convert header dictionary to a pandas Series.
\begin{quote}\begin{description}
\item[{Returns}] \leavevmode
\begin{itemize}
\item {} 
\sphinxstylestrong{cifheader\_series} (\sphinxstyleemphasis{pandas Series containing all information})

\item {} 
\sphinxstyleemphasis{stored on the header dictionary.}

\end{itemize}


\end{description}\end{quote}

\end{fulllineitems}

\index{parse() (xtalphases.data.preprocess.CIFParser method)@\spxentry{parse()}\spxextra{xtalphases.data.preprocess.CIFParser method}}

\begin{fulllineitems}
\phantomsection\label{\detokenize{usage/installation:xtalphases.data.preprocess.CIFParser.parse}}\pysiglinewithargsret{\sphinxbfcode{\sphinxupquote{parse}}}{\emph{integer\_indexes={[}'index\_h', 'index\_l', 'index\_k', 'pdbx\_r\_free\_flag'{]}}}{}
Parse both header and reflections.

\end{fulllineitems}

\index{reflections\_parse() (xtalphases.data.preprocess.CIFParser method)@\spxentry{reflections\_parse()}\spxextra{xtalphases.data.preprocess.CIFParser method}}

\begin{fulllineitems}
\phantomsection\label{\detokenize{usage/installation:xtalphases.data.preprocess.CIFParser.reflections_parse}}\pysiglinewithargsret{\sphinxbfcode{\sphinxupquote{reflections\_parse}}}{\emph{integer\_columns={[}'index\_h', 'index\_l', 'index\_k', 'pdbx\_r\_free\_flag'{]}}}{}
Parse reflections on .cif file.
\begin{quote}\begin{description}
\item[{Parameters}] \leavevmode\begin{itemize}
\item {} 
\sphinxstyleliteralstrong{\sphinxupquote{integer\_indexes}} (\sphinxstyleliteralemphasis{\sphinxupquote{reflection columns containing}}) \textendash{} 

\item {} 
\sphinxstyleliteralstrong{\sphinxupquote{entries.}} (\sphinxstyleliteralemphasis{\sphinxupquote{integer}}) \textendash{} 

\end{itemize}

\end{description}\end{quote}

\end{fulllineitems}

\index{reflections\_to\_df() (xtalphases.data.preprocess.CIFParser method)@\spxentry{reflections\_to\_df()}\spxextra{xtalphases.data.preprocess.CIFParser method}}

\begin{fulllineitems}
\phantomsection\label{\detokenize{usage/installation:xtalphases.data.preprocess.CIFParser.reflections_to_df}}\pysiglinewithargsret{\sphinxbfcode{\sphinxupquote{reflections\_to\_df}}}{\emph{columns=None}}{}
Convert reflection dictionary to a pandas DataFrame.
\begin{quote}\begin{description}
\item[{Returns}] \leavevmode
\begin{itemize}
\item {} 
\sphinxstylestrong{refln\_df} (\sphinxstyleemphasis{pandas DataFrame containing all information})

\item {} 
\sphinxstyleemphasis{stored on the reflection dictionary.}

\end{itemize}


\end{description}\end{quote}

\end{fulllineitems}


\end{fulllineitems}

\index{FaradayRotation (class in astroph.random\_code)@\spxentry{FaradayRotation}\spxextra{class in astroph.random\_code}}

\begin{fulllineitems}
\phantomsection\label{\detokenize{usage/installation:astroph.random_code.FaradayRotation}}\pysiglinewithargsret{\sphinxbfcode{\sphinxupquote{class }}\sphinxcode{\sphinxupquote{astroph.random\_code.}}\sphinxbfcode{\sphinxupquote{FaradayRotation}}}{\emph{polarization\_angles}}{}
A class for Faraday Rotation measurements to determine
interstellar magnetic fields.
\begin{quote}\begin{description}
\item[{Parameters}] \leavevmode
\sphinxstyleliteralstrong{\sphinxupquote{polarization\_angles}} (\sphinxstyleliteralemphasis{\sphinxupquote{array}}) \textendash{} Polarization angle measurements.

\end{description}\end{quote}

\end{fulllineitems}

\index{\_get\_magnetic\_field\_map() (in module astroph.random\_code.FaradayRotation)@\spxentry{\_get\_magnetic\_field\_map()}\spxextra{in module astroph.random\_code.FaradayRotation}}

\begin{fulllineitems}
\phantomsection\label{\detokenize{usage/installation:astroph.random_code.FaradayRotation._get_magnetic_field_map}}\pysiglinewithargsret{\sphinxcode{\sphinxupquote{astroph.random\_code.FaradayRotation.}}\sphinxbfcode{\sphinxupquote{\_get\_magnetic\_field\_map}}}{\emph{self}}{}
Testing math docstring.
\begin{quote}\begin{description}
\item[{Returns}] \leavevmode
\sphinxstylestrong{map} \textendash{} Array representing the magnetic field map. Here
\(\alpha < \beta\).

\item[{Return type}] \leavevmode
array

\end{description}\end{quote}
\begin{equation*}
\begin{split}\mathbf{B} = \frac{\mu_0}{4\pi}\end{split}
\end{equation*}
\begin{sphinxVerbatim}[commandchars=\\\{\}]
\PYG{g+gp}{\PYGZgt{}\PYGZgt{}\PYGZgt{} }\PYG{k+kn}{import} \PYG{n+nn}{math}
\PYG{g+gp}{\PYGZgt{}\PYGZgt{}\PYGZgt{} }\PYG{n}{math}\PYG{o}{.}\PYG{n}{sqrt}\PYG{p}{(}\PYG{l+m+mf}{2.}\PYG{p}{)}
\PYG{g+go}{1.4142135623730951}
\end{sphinxVerbatim}

\end{fulllineitems}



\chapter{Indices and tables}
\label{\detokenize{index:indices-and-tables}}\begin{itemize}
\item {} 
\DUrole{xref,std,std-ref}{genindex}

\item {} 
\DUrole{xref,std,std-ref}{modindex}

\item {} 
\DUrole{xref,std,std-ref}{search}

\end{itemize}



\renewcommand{\indexname}{Index}
\printindex
\end{document}